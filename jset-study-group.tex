\documentclass[dvipdfmx]{jset-study-group}

\usepackage{graphicx}
\usepackage{amssymb} % 記号

% 表関連
\usepackage{colortbl,array,xcolor}
\usepackage{tabularx}
\newcolumntype{C}{>{\centering}X}
\renewcommand{\tabularxcolumn}[1]{m{#1}}
\usepackage{booktabs}

% 箇条書き
\usepackage{enumitem}

\title{日本教育工学会研究会の原稿見本}
\etitle{How to Write Your Paper for the JSET Research Reports}
\subtitle{研究報告集の公開に向けて}
\esubtitle{Towards Open Access}
\abstract{日本教育工学会は,年間5回の研究会を開催している.
発表原稿は,「日本教育工学会研究報告集」として出版される.
原稿はこの原稿見本に従って執筆することが求められる.
書式が守られていない場合,投稿後に修正依頼がある.
原稿は,研究会Webページの「原稿投稿申込」より,発表申込時に発行された「受付キー」を使用して電子投稿する$^{1)}$.
印刷時のイメージ崩れを防ぐため,PDF形式のみ受け付ける.}
\keyword{
	高等教育\hspace{1zw}
	遠隔教育\hspace{1zw}
	情報検索\hspace{1zw}
	マルチメディア教材\hspace{1zw}
	原稿仕様\hspace{1zw}
	可読性
}

\author{日本 太郎$^\ast$\hspace{1cm}教育 次郎$^\ast$\hspace{1cm}工学 花子$^{\ast\ast}$}
\eauthor{Taro Nihon$^\ast$\hspace{1cm}Jiro Kyouiku$^\ast$\hspace{1cm}Hanako Kougaku$^{\ast\ast}$}

\affiliation{日本教育大学工学部$^\ast$\hspace{5mm}日本教育大学教育学部$^{\ast\ast}$}
\eaffiliation{
Faculty of Engineering, Nihon Kyouiku University$^\ast$\\
Faculty of Education, Nihon Kougaku University$^{\ast\ast}$}

% 下線
\usepackage{ulinej}

% レイアウトの確認
%\usepackage{layout}

\begin{document}

%\layout

\maketitle

\section{はじめに}

日本教育工学会研究会では.教育工学に関する研究を発表することができる.
\ulinej{ただし,提出された原稿が本原稿見本と著しく異なるものであると研究会委員会が判断した場合は,修正して次回以降の研究会に改めて投稿して頂くことをお願いすることがある.}

\section{用紙と原稿枚数}

原稿は,A4サイズ縦の用紙を使用する.1枚が1ページとする.原稿枚数は,\ulinej{4枚~8枚}の偶数組みである.
万一,奇数になった場合には,事務局にて白紙ページを1枚追加して偶数ページに揃える.
なお,原稿見本に従って執筆した場合に\ulinej{9枚以上または3枚以下となる場合は受理できない}.
\ulinej{ページ番号は付けない.}

そのままオフセット印刷するので,\ulinej{完成原稿の状態で提出}する.
印刷においてはサイズを縮小し,\ulinej{B5判サイズ白黒}となる.
投稿前に刷り上がり時の可読性を確認しておくこと.また\ulinej{白黒で出力したPDF}を提出する$^{2)}$.

\section{原稿の書き方}

この原稿見本は,日本教育工学会研究報告集の原稿書式・体裁を統一することを目的とする.
Microsoft Wordを用いるなら,本稿を書き換えて執筆できる.

\subsection{題名・著者名・所属機関}

原稿の冒頭には,和文と英文で題名,著者名,所属機関を入れる.

和文題目は16ptのMSゴシック$^{3)}$とする.
論文等の内容が明確に分かるような簡潔なものにする.
英文題目は和文題目の次の行にCentury の11ptとし,助詞等を除く各単語の先頭文字は大文字とする.

副題はない方が望ましい.
副題がある場合は,和文副題は12ptのMS明朝とし,英文題目から1行分あけて記述する.
また,副題であることが分かるよう,- -で囲う$^{4)}$.
英文副題は和文副題の次の行にCentury の10ptとし,\ulinej{助詞等を除く各単語の先頭文字は大文字とする.}

和文著者名は,12ptのMS明朝とし,英文題目又は英文副題から1行分あけて記述する.著者が複数で,その所属が別の場合には,*印を付して対応させる.\ulinej{所属が4カ所以上の場合}は,*ではなく日本 太郎$^1$のように上付き半角数字で表記する.なお,単著や単一組織の場合は,*印は必要ない.英文著者名は和文著者名の次の行にCentury 10ptとし,First nameを最初に記述する.先頭文字のみ大文字とする.

和文所属は,12ptのMS明朝とし,英文著者名から1行あけて記述する.著者名に対応した所属に同じ*印を付す.所属は表1に従って機関名と組織名を記す.大学は法人格を省略する.キャンパス名や研究室名等は表記しない.初等中等教育機関は設置主体を明記する.大学を除く法人は法人格を明記する.法人格を有しない任意団体,弁護士等を含む自由業等個人の資格で発表する場合は所属を記載しない.所属は研究当時のものではなく,発表時点のものとする.「元○○」といった表記にはしない.英文所属は和文所属の次の行にCenturyの10ptとして記述する.

表1に当てはまらず,所属の書き方の判断に迷う場合は,研究会事務局に問い合わせてから原稿の執筆を始めること.なお,研究報告集の目次等には,表1の組織名を省略し,機関名のみ記載する.

表1 機関名・組織名の記載例

機関名
組織名
大学学生
○○大学
××学部△△学科
○○大学大学院
××研究科△△専攻
大学教職員
○○大学
××学部
○○大学大学院
××研究科

○○大学
※ただし機関名のみを扱う場合は「○○大学学術院」「○○大学研究院」とする
××学術院
××研究院

○○大学
××センター

○○大学附属図書館


○○大学附属病院

初等中等教育
○○大学附属××学校
△△科
○○立教育センター


○○教育委員会

○○立××学校
△△科
法人
株式会社○○
○○株式会社

○○株式会社××研究所


公益財団法人○○

特定非営利活動法人○○

その他
(所属を記載しない)
※法人格を有しない任意団体,退職後上記機関等に属さない場合等


\subsection{あらましとキーワード}

あらましは,はじめに<あらまし>と記載し,その後全角スペースを1文字入れた後に200 字程度にまとめる.あらましは英文所属から1行分あけて記述する.両端から全角1文字分インデントを行う.
キーワードは,はじめに<キーワード>と記載し,その後全角スペースを1文字入れた後に5~6語程度を,全角スペースで区切って並べる.キーワードが2行にわたる場合,1つの単語が2行にわたって表記されないようにする.キーワードはあらましから1行分あけて記述する.両端から全角1文字分インデントを行う.

\subsection{本文}

各段落の冒頭は,全角1文字分下げて書き始める.句読点はコンマ(,)とピリオド(.)を用いる.

本文は,10ptのMS明朝(英数字はCentury)を用いて執筆する.日本語はプロポーショナルフォントを用いない.記述は常用漢字,現代かなづかいとし,簡潔かつ明瞭にする.読者の多様な専門的背景を念頭におき,記述形式に注意する5).

数字は算用数字を使用する.1桁数字は全角数字(ただし,章・節・項の番号は半角数字),2桁以上の数字は半角数字にて表記する.

英語の略語表記は,初出時にはCAI (Computer Assisted Instruction)のように,略さない表記をする.英語の略語表記は,英文題目には用いない(キーワードに用いた場合には,本文中初出時に略さない表記をする).

データの基本統計量(例.SD,F,r,pなど)は斜体にて表記する.

\subsubsection{章見出し・節見出し・項見出し}

各章・節の見出しの前に,前の章・節の最後の行との間に1行空白行を設ける.ただし,章や節がページや段組の最初から始まる場合等や,項の見出しはこの限りではない.また,見出しは段の最下部からは始めない.その場合,次の段またはページから始める.

章見出しは10.5pt,節見出しと項見出しは10ptのMSゴシック(英数字もMSゴシックとする)を用いる.見出しに下線等を付さない.いずれの見出しも,1字下げや中央揃えはせず,行の左端に寄せる.

見出しの番号は,いずれも半角英数字にて記載する.1.より連番となるようにして,2.3.1.のように付す.最後は「.」で終わる.

\subsection{図表・写真}

図と表および写真は,完成原稿のイメージとなるように適切な箇所に貼り付ける.図表を最後にまとめ,文中に図の挿入位置を示すといった方法にはしない.図表が大きいなどの場合,段組をまたいでもよいが,ページはまたがないように工夫する.図表は本文との差異がわかるように,前後に余白をとる.

いずれもB5版に縮小印刷されることを勘案して,十分な大きさ・明瞭さを持つように設定する.また白黒で印刷されるので,色で区別するような表現には注意が必要である6).

各図表・写真には,図1,表1,写真1のように,それぞれ一連番号を付け,タイトルを付す.図・写真は直下に,表は直上に,中央揃えで表記する.本学会の印刷された論文誌にならい,「図」「表」「写真」の部分と番号はMSゴシックの10pt,タイトルはMS明朝またはCenturyの10ptにて表記する.図表・写真番号とタイトルの間は全角スペースとし,「:」などは使わない.

本文中で図表・写真を参照するときには,以下のように表記する.

図1は………  表1は………

………となっている(写真1).

以上のように,キャプションと同様に「図」「表」「写真」の部分と番号はゴシックにて表記する.また,番号については,1桁なら全角,2桁以上なら半角で示す.

図1 図のサンプル



写真1 写真のサンプル

表2 表のサンプル
人数
引用表記
1人
(教育 2008)および
(Kyouiku 2008)
2人
(教育・工学 2008)および
(Kyouiku and Kougaku 1992)
3人以上
(教育ほか 2008)および
(Kyouiku et al. 2008)

\section{注および謝辞}

注はできるだけ少なくする.必要な場合,参考文献の前に一括して入れ,本文中の該当箇所の右肩に上付きの半角英数字で1),2)のように示す.本文の後は,注,謝辞,参考文献の順番になるようにする.
注や謝辞の見出しは,MSゴシックの10.5ptとし,左寄せにして表記する.見出しの前には1行分の空行をあける.内容はMS明朝またはCenturyの10ptを用いる.


\section{参考文献}

\subsection{本文中での参考文献の引用}

本文中において参考文献を引用する場合,次のように表記し,番号などによる引用表記([1][2]…や(1)(2)…)は,用いず,以下のように記述する.
Kyouiku(2008a)は………
教育(2008)は………
………といっている(Kyouiku 2008b).
………といっている(工学 2008).
カッコ内の著者名と発表年の間は,半角スペースで1文字あける.間にカンマ(,)は入れない.英文著者名はFamily Nameの最初の1文字を半角英字大文字で,残りを小文字で表記する.
著者人数によって,表2のように表記する.複数著者の場合,日本語と英語で著者間の区切りが点(・)かandかが変わるので注意する.複数の引用をする場合,次のようにする.(教育・工学 2008,Kyouiku et al. 2008)

\subsection{参考文献リストの記述形式}

参考文献は,原稿の最後に著者苗字のアルファベット順で一括する(和文誌・英文誌で分けない).本文中で引用あるいは参照している文献のみをここに挙げる.
参考文献の見出しは,左寄せし,10.5ptのMSゴシックにて表す.章・節の見出しと同様,見出しの前に,1行分の空行をあける.内容には10ptのMS明朝(英数字はCentury)を用いる.
雑誌の場合,著者,発表年,表題,雑誌名,巻数,号数,論文所在ページの順とする.英語で表記する英文の雑誌名は斜体にて表記する.
書籍の場合,著者,発行年,書名,発行所,発行地,(ページを入れる場合はページ)の順とする.英語で表記する英文の書籍名は斜体にて表記する
URL (Uniform Resource Locator)を参照する場合は,著者,発行年,表題,URL,参照日の順とする.URLはワープロソフトの機能により自動的にハイパーリンクが付されることがあるが,ハイパーリンクを削除するか,アンダーラインを表示させないようにする.
著者名は,日本語・漢字・ハングル文字で表記する場合は,該当の著作物の著者の姓と名の両方を表記する.姓名の間にカンマ(,)は不要である.複数の著者の場合は,ひとりずつ全角カンマ(,)で区切って全員の氏名を列挙する.それ以外の語種で著者名を表記する場合は,Family Name(最初の1文字のみ大文字)とFirst Nameのイニシャル(大文字)で表す.複数の著者の場合は,半角カンマと半角スペース(, )で区切り,最後だけandで区切る(2名の場合はandだけである).
同一著者の著作物を複数扱う場合,発表年の昇順で列挙する.全く同一の著者が同一年に複数の文献を発表したものを参照する場合のみ,発表年の表記は,2008a,2008bのようにa, b, c, …を付して表記する.本文中で参照する際にも同様にa, b, c, …を付して,(Kyouiku 2008a, Kyouiku 2008b)として,同定可能にする.複数の文献を参照する場合,日本語・漢字・ハングル文字の文献名のみの場合は全角カンマ(,)で,その他の場合は半角カンマと半角スペース(, )で区切る.
巻号・ページ数の表記については,学術論文誌の場合は,巻数をゴシック体で表し,そのあとにCenturyもしくはMS明朝で号数,続いてコロン(:)とページ数をそのまま表記する.この場合は,「pp.」の表示は必要ない.一方,学会の大会講演論文集や研究会報告集等の場合は,例えばVol.32, No.2, pp.143-145のように,巻号とページをVol.,No.,pp.でそれぞれ示し,カンマ(,)で区切る.なお,学会によってはVol.,No.以外の表記方法をとっている合もあるので注意すること(例えば,日本教育工学会研究報告集や,情報処理学会研究報告などについては,Vol., No.を表記しないこともある).開始ページ番号と終了ページ番号の間は,半角-でつなぐ.
1つの文献情報が2行以上にわたる場合,2行目以降は全角2文字分インデントする.

\section{注}

1) 原稿の電子メールでの提出は,原則として受け付けない.原稿形式などの問い合わせは,研究会事務局へ連絡する.学会本部事務局は,研究会発表に関する問い合わせには回答できない.
2) PDF作成においては,図表や画像の品質が極端に落ちないように,PDF出力のプロパティで,「白黒」「プレス品質」等に設定する.
なお,PDF作成において,図表中の文字が化けたり,ページ中の文字が抜け落ちたりする現象が確認されている.PDF作成後,内容を確認すること(最終原稿の場合,作業工程上,研究会事務局のチェックは困難である).
提出されたPDFファイルは事務局にて印刷や編集を行うので,パスワードなどによるセキュリティ設定はしない.
3) MS系のフォントの利用が困難な場合は,それに準じるものを利用する.
4) 題名が2行以上にわたる場合,Microsoft Wordでは,行を変える際にShiftキーと改行キーを同時に押すことで,行間を詰めて題目を記載できる.
5) 形式に関する注意ではないが,原稿は,他者の著作権や,研究に関わる個人・集団(研究対象となった個人・集団や研究に関連のある個人・集団)のプライバシーや名誉に関する十分な配慮のもとに執筆すること. 
6) Microsoft Office 2007では,Excelで生成したグラフに白黒系の塗りつぶしパターンを設定してWordに貼り付けることが難しい.ネット上で配布されているアドインを利用するなどが必要となる(利用は自己責任).

\bibliographystyle{sieicej}

\bibliography{sample-bib}

\end{document}  
